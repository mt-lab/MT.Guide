\part[Docs]{Шаблоны документов}
\chapter[Hardware_Doc]{Документация на изделия}
\section[Необходимые_разделы]{Разделы документации на прибор}

\chapter[TZ]{ТЗ}
\section[TZ_Parts]{Разделы ТЗ}
Без ТЗ, как известно, результат ХЗ. Нет смысла начинать работу над чем угодно, если у всех разные представления об этом чём-то. Сначала, в общих чертах, может казаться, что все отлично друг друга понимают, но ощущение это, скорей всего, обменачиво. Именно для разрешения возможных болезненных разногласий на более поздних этапах разработки/подготовки и стоит адекватно составлять Техническое задание.

Брать ли за составление этого документа деньги - решать по обоюдному согласию с заказчиком. Понятно, что глупо и жадно брать плату за пару уточняющих вопросов, если человек и так хорошо понимает, что ему нужно, а для мелких деталей ему не хватает квалификации.

Однако, требование оплаты на данном этапе позволяет отсечь желание связываться с нами у фантазёров и людей, которые не готовы серьёзно отнестись к своим идеям, а хотят лишь напыщенного общения на околотехнические темы, безжалостно выбрасывая своё и наше время.