\documentclass[a5paper,14pt]{extbook}

%% Шрифты и кириллица
\usepackage{fontspec}
\usepackage{polyglossia}
\defaultfontfeatures{Ligatures=TeX}
\setdefaultlanguage{russian}
\setotherlanguage{english}
\setmainfont{Courier New}
\newfontfamily{\latinfont}{Courier New}
\newfontfamily{\cyrillicfont}{Courier New}
\newfontfamily{\cyrillicfonttt}{Courier New}

\usepackage{cite,float}

%% Общее оформление
\usepackage{indentfirst}%добавляем отступ первой строки абзаца
\setlength{\parindent}{1.5cm}%задаём величину отступа

%% Меняем поля страницы
\usepackage[hscale=0.9, vscale=0.8]{geometry}

%% Работа с картинками
\usepackage{graphicx}
\graphicspath{{./src/images/}}

%% Перечисления
\renewcommand{\theenumi}{\arabic{enumi}}% Меняем везде перечисления на цифра.цифра
\renewcommand{\labelenumi}{\arabic{enumi}}% Меняем везде перечисления на цифра.цифра
\renewcommand{\theenumii}{.\arabic{enumii}}% Меняем везде перечисления на цифра.цифра
\renewcommand{\labelenumii}{\arabic{enumi}.\arabic{enumii}.}% Меняем везде перечисления на цифра.цифра
\renewcommand{\theenumiii}{.\arabic{enumiii}}% Меняем везде перечисления на цифра.цифра
\renewcommand{\labelenumiii}{\arabic{enumi}.\arabic{enumii}.\arabic{enumiii}.}% Меняем везде перечисления на цифра.цифра

%% Работа с фигурами и цветом
\usepackage{caption}
\usepackage{color}
\usepackage{xcolor}
\newcommand{\imgh}[3]
{
\begin{figure}[h]
    \center{\includegraphics[width=#1]{#2}}
    \caption{#3}
    \label{ris:#2}
\end{figure}
}

% Listing environment
\usepackage[outputdir=auxiliary]{minted}
